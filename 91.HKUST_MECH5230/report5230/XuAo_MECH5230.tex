%!Mode:: "TeX:UTF-8"
%----------------------------------------------------------------------------------------------------------------------------------------------------------
\documentclass[12pt,a4paper]{article} % Default font size is 12pt, it can be changed here
\usepackage[top=33mm, bottom=33mm]{geometry}
\usepackage{graphicx,subfigure} % Required for including pictures
\usepackage{float} % Allows putting an [H] in \begin{figure} to specify the exact location of the figure
\usepackage{wrapfig} % Allows in-line images
\usepackage{amsmath,amsthm,amsfonts,amssymb,bm} %数学公式支持宏包
\usepackage{CJK,CJKnumb,CJKulem}         % 中文支持宏包
\bibliographystyle{unsrt}
\linespread{1.2} % Line spacing
\setlength\parindent{0pt} % Uncomment to remove all indentation from paragraphs
\graphicspath{{./images/}} % Specifies the directory where pictures are stored
\usepackage{fancyhdr} % Custom headers and footers
\pagestyle{fancyplain} % Makes all pages in the document conform to the custom headers and footers
%\fancyhead{} %\fancyfoot{} No page header OR footer - if you want one, create it as demo
\fancyhead[L]{MECH5230}
\fancyhead[C]{Name: \textbf{Xu Ao}}
\fancyhead[R]{ID: \textbf{20153308}}
\fancyfoot[L]{} % Empty left footer
\fancyfoot[C]{\thepage} % Page numbering for right footer
\fancyfoot[R]{}
\renewcommand{\headrulewidth}{2pt} % Remove header underlines
\renewcommand{\footrulewidth}{0pt} % Remove footer underlines

%---------------List codes-----------------------------
%---------------List codes-----------------------------
% packages
\RequirePackage{color}          % color image and color definition
\RequirePackage{fancyhdr,fancybox}       % personalize page header
\RequirePackage{graphicx}       % include graphics
\RequirePackage{ifthen}         % logic options
\RequirePackage{lastpage}       % refer to the number of last page
\RequirePackage{listings}       % format source code
%
\definecolor{grey}{rgb}{0.8,0.8,0.8}
\definecolor{darkgreen}{rgb}{0,0.3,0}
\definecolor{darkblue}{rgb}{0,0,0.3}
\def\lstbasicfont{\fontfamily{pcr}\selectfont\footnotesize}
\lstset{%
% indexing
    numbers=left,
    numberstyle=\small,%
% character display
    showstringspaces=false,
    showspaces=false,%
    tabsize=4,%
% style
    frame=lines,% OR single %
    basicstyle={\footnotesize\lstbasicfont},%
    keywordstyle=\color{blue}\bfseries,%
    commentstyle=\color[cmyk]{1,0,1,0},
    stringstyle=\color{green}
    identifierstyle=\color{blue},
}
\lstset{breaklines}
\lstloadlanguages{Fortran,C,C++,Matlab,Mathematica}
%---------------List codes-----------------------------
%---------------List codes-----------------------------

%----------------------------------------------------------------------------------------------------------------------------------------------------------
\begin{document}
\begin{CJK}{UTF8}{song} %开始中文环境
%----------------------------------------------------------------------------------------------------------------------------------------------------------
\title{MECH5230 Homework Assignment}
\maketitle
\tableofcontents % Include a table of contents
\newpage % Begins the essay on a new page instead of on the same page as the table of contents
%----------------------------------------------------------------------------------------------------------------------------------------------------------






\section{1D Time-dependent Heat Equation}
Solve the time-dependent heat equation $T_{t}=\alpha T_{xx}$, $0 \leq x \leq L$.

Initial condition: $T(x,0)=c \sin(\pi x/L)$.

Boundary condition: $T(0,t)=T(L,t)=0$.

Exact solution is $T_{exact}(x,t)=c \exp (-\alpha \pi ^{2} t /L^{2})\sin (\pi x /L)$

Let $c=100^{\circ}C$, $L=1m$, $\alpha=0.02 m^{2}/h$.
\begin{enumerate}
  \item Simple explicit method.

Forward Time Central Space (FTCS):
\begin{equation}
\frac{T^{n+1}_{i}-T^{n}_{i}}{\Delta t}=\alpha \frac{T_{i+1}^{n}-2T_{i}^{n}+T_{i-1}^{n}}{(\Delta x)^{2}}
\end{equation}
i.e.,
\begin{equation}
T^{n+1}_{i}=T^{n}_{i}+\frac{\alpha \Delta t}{(\Delta x)^{2}}(T_{i+1}^{n}-2T_{i}^{n}+T_{i-1}^{n})
\end{equation}

  \item Alternating direction explicit (ADE) method proposed by Barakat and Clark.

the calculation procedure is simultaneously marched in both directions
($p_{i}^{n+1}$ is updated from left-side points, while $q_{i}^{n+1}$ is updated from right-side points)
the resulting solution are averaged to obtain the final value of $T_{i}^{n+1}$.
\begin{equation}
    \begin{split}
& \frac{p^{n+1}_{i}-p^{n}_{i}}{\Delta t}=\alpha \frac{p_{i+1}^{n}-p_{i}^{n}-p_{i}^{n+1}+p_{i-1}^{n+1}}{(\Delta x)^{2}} \\
& \frac{q^{n+1}_{i}-q^{n}_{i}}{\Delta t}=\alpha \frac{q_{i+1}^{n+1}-q_{i}^{n+1}-q_{i}^{n}+q_{i-1}^{n}}{(\Delta x)^{2}} \\
& T_{i}^{n+1} = \frac{1}{2}(p^{n+1}_{i}+q^{n+1}_{i}) \\
    \end{split}
\end{equation}
i.e.,
\begin{equation}
    \begin{split}
& p_{i}^{n+1}[1+\frac{\alpha \Delta t}{(\Delta x)^{2}}]=\frac{\alpha \Delta t}{(\Delta x)^{2}}p_{i-1}^{n+1}+[1-\frac{\alpha \Delta t}{(\Delta x)^{2}}]p_{i}^{n}+\frac{\alpha \Delta t}{(\Delta x)^{2}}p_{i+1}^{n} \\
& q_{i}^{n+1}[1+\frac{\alpha \Delta t}{(\Delta x)^{2}}]=\frac{\alpha \Delta t}{(\Delta x)^{2}}q_{i+1}^{n+1}+[1-\frac{\alpha \Delta t}{(\Delta x)^{2}}]q_{i}^{n}+\frac{\alpha \Delta t}{(\Delta x)^{2}}q_{i-1}^{n} \\
& T_{i}^{n+1} = \frac{1}{2}(p^{n+1}_{i}+q^{n+1}_{i}) \\
    \end{split}
\end{equation}

\end{enumerate}

\subsection{Comparison of FTCS and ADE Scheme}
For $\Delta x=0.1$ and $\Delta t=0.1$ compare results for $t=10h$ in a graph, as shown in Fig.\ref{Fig.1-a}

\subsection{Accuracy of FTCS Scheme}

Refining the mesh as $\Delta x=0.066667$ while remaining the time step as $\Delta t=0.1h$ compare results for $t=10h$ in a graph, as shown in Fig.\ref{Fig.1-b}

\begin{figure}[htbp]
\centering
\subfigure[$\Delta x=0.1, \ \Delta t=0.1$]
{
  \includegraphics[width=7cm]{./1/a.png} \label{Fig.1-a}
}
\subfigure[$\Delta x=0.066667, \ \Delta t=0.1$]
{
  \includegraphics[width=7cm]{./1/b.png}  \label{Fig.1-b}
}
\caption{Comparison results of FTCS (Forward Time Central Space) and ADE (Alternating Direction Explicit) scheme ($t=10h$) with different mesh resolution}
\end{figure}


For FTCS scheme, after refining the mesh, however, the error does \textbf{NOT} reduce as suggested by $O(\Delta x^{2})$.
This is mainly due to the reason that for FTCS scheme,
the error is related with both space step and time step, i.e., $O(\Delta x^{2},\Delta t)$.
The influence of time step $\Delta t$ on error is prominent for time-dependent problem.
Later we will found different situation in Problem 2(2), where a steady problem is considered.

\subsection{Stability of FTCS Scheme}
For FTCS scheme, the error is $O(\Delta x^{2},\Delta t)$. At given mesh resolution ($\Delta x$), the error will be dependent on $\Delta t$.

Define CFL number as
\begin{equation}
CFL=\frac{\alpha \Delta t}{(\Delta x)^{2}}
\end{equation}
by choose smaller $\Delta t$ (i.e., choose smaller $CFL$ number), we can actually reduce the error.

Here, we choose $\Delta x=0.1$ and $CFL=0.06, 0.6$.

Define $L^{2}$ norm error of temperature as
\begin{equation}
Error(t)=\sqrt{\frac{\sum[T_{numerical}(t)-T_{exact}(t)]^{2}}{\sum T_{exact}(t)^{2}}}
\end{equation}

We can found from Fig.\ref{Fig.1-error} that $L^{2}$ norm error of temperature for $CFL=0.6$ progressively become larger,
while slightly change for $CFL=0.06$.

\begin{figure}[htbp]
\centering
\includegraphics[width=7cm]{./1/error.png}
\caption{Error evolution within $20h$ of problem time under different $CFL$ number for FTCS method} \label{Fig.1-error}
\end{figure}

%%%\begin{figure}[htbp]
%%%\centering
%%%\subfigure[Temperature distribution at $t=20h$]
%%%{
%%%        \includegraphics[width=7cm]{./1/cfl.png}
%%%}
%%%\subfigure[centerline temperature vs. time]
%%%{
%%%        \includegraphics[width=7cm]{./1/cfl_time.png}
%%%}
%%%\caption{Comparison FTCS results of FTCS under different CFL number}
%%%\end{figure}
%%%We can found that $CFL=0.01$ (the green line) has the smallest error among the test cases.

\subsection{Accuracy of ADE Scheme}
Again, define CFL number as
\begin{equation}
CFL=\frac{\alpha \Delta t}{(\Delta x)^{2}}
\end{equation}

Choose $\Delta x=0.1$ and $CFL=1, 2, 3$, compare corresponding results with exact solution as shown in Fig.\ref{Fig.1-d}.
\begin{figure}[htbp]
\centering
{
    \includegraphics[width=9cm]{./1/d.png}
}
\caption{Comparison ADE results of FTCS under different CFL number} \label{Fig.1-d}
\end{figure}
We can found, as $CFL$ number increase, the discrepancy becomes noticeable.


\newpage
\section{1D Steady-state Heat Equation with Source Term}

Solve heat equation with a source term $T_{t}=\alpha T_{xx} +S(x)$, $0 \leq x \leq L_{x}$.

Initial condition: $T(x,0)=0$.

Boundary condition: $T(0,t)=0,\ T(L_{x},t)=T_{steady}(L_{x})$.

Exact steady solution is $T_{steady}(x)=x^{2}e^{-x}$.

Let $\alpha=1,\ L_{x}=15,\ S(x)=-(x^{2}-4x+2)e^{-x}$.

\subsection{Exact Solution}
Verify the exact steady solution is $T_{steady}(x)=x^{2}e^{-x}$.

For steady solution, $T_{t}=0$, then the heat equation at steady state is
\begin{equation}
\alpha T_{xx} +S(x) = 0
\end{equation}

Since,
\begin{equation}
    \begin{split}
    & S(x)=-(x^{2}-4x+2)e^{-x} \\
    & T(x) = x^{2} e^{-x} \\
    & T_{x}(x) = 2xe^{-x}-x^{2}e^{-x}=(2x-x^{2})e^{-x} \\
    & T_{xx}(x) = (2-2x)e^{-x}-(2x-x^{2})e^{-x} = (2-4x+x^{2})e^{-x} \\
    \end{split}
\end{equation}
thus,
\begin{equation}
\alpha T_{xx}(x)+S(x)=(2-4x+x^{2})e^{-x}-(x^{2}-4x+2)e^{-x} =0
\end{equation}
proving that exact solution is $T_{steady}(x) = x^{2} e^{-x}$

Plot exact solution as shown in Fig.\ref{Fig.2-a}.
\begin{figure}[htbp]
\centering
{
    \includegraphics[width=7cm]{./2/a.png}
}
\caption{Exact solution of 1D heat equation with source term} \label{Fig.2-a}
\end{figure}

\subsection{FTCS Scheme}
Explicit Euler for time advancement and second-order central difference scheme on a uniform grid:
\begin{equation}
X_{i}=i\cdot \ \Delta x
\end{equation}
\begin{equation}
\frac{T_{i}^{n+1}-T_{i}^{n}}{\Delta t}=\alpha \frac{T_{i+1}^{n}-2T_{i}^{n}+T_{i-1}^{n}}{(\Delta x)^{2}}+S(X_{i})
\end{equation}
i.e.,
\begin{equation}
T_{i}^{n+1}=T_{i}^{n}+\frac{\alpha \Delta t}{(\Delta x)^{2}}(T_{i+1}^{n}-2T_{i}^{n}+T_{i-1}^{n})-\Delta t (X_{i}^{2}-4X_{i}+2)e^{-X{i}}
\end{equation}
Plot the exact and numerical steady solutions for $N_{x}=10,20$ as Fig.\ref{Fig.2-b}.

\begin{figure}[htbp]
\centering
{
    \includegraphics[width=9cm]{./2/b.png}
}
\caption{Results of FTCS with different uniform mesh grids} \label{Fig.2-b}
\end{figure}
We can found that increase the grid numbers indeed reduce the error.
It should be noted that different from Problem 1.(2), which is time-dependent problem, the error is $O(\Delta x^{2},\Delta t)$.
While for steady problem, like this one, $\Delta t$ have little influence on the error.
Small $\Delta t$ requires longer iterative steps to converge to steady solution, while larger $\Delta t$ requires less (so long the program is stable). The influence of space step $\Delta x$ on error is prominent for steady problem. Thus, refine the mesh resolution reduces the error.

\subsection{Non-uniform Grid}
On a non-uniform grid $X_{j}=L_{x}[1-\cos(\frac{\pi j}{2N_{x}})]$, perform Taylor expansion,
\begin{equation}
    \begin{split}
& f_{j+1}=f_{j}+\Delta x_{j+1} f_{j}^{'}+\frac{(\Delta x_{j+1})^{2}}{2}f_{j}^{''}+\frac{(\Delta x_{j+1})^{3}}{3!}f_{i}^{'''} + \cdots \\
& f_{j-1}=f_{j}-\Delta x_{j} f_{j}^{'}+\frac{(\Delta x_{j})^{2}}{2}f_{j}^{''}-\frac{(\Delta x_{j})^{3}}{3!}f_{i}^{'''} + \cdots \\
    \end{split}
\end{equation}
thus, we have
\begin{equation}
f_{j}^{''}=2\ \frac{\Delta x_{j}(f_{j+1}-f_{j})-\Delta x_{j+1}(f_{j}-f_{j-1})}{\Delta x_{j} \ \Delta x_{j+1} \ (\Delta x_{j}+\Delta x_{j+1})}
\end{equation}
where $\Delta x_{j}=X_{j}-X_{j-1}$.

Discretization of heat equation leads to
\begin{equation}
\begin{split}
& \frac{T_{j}^{n+1}-T_{j}^{n}}{\Delta t}=\alpha \ 2 \frac{\Delta x_{j}(T_{j+1}^{n}-T_{j}^{n})-\Delta x_{j+1}(T_{j}^{n}-T_{j-1}^{n})}{\Delta x_{j} \ \Delta x_{j+1} \ (\Delta x_{j}+\Delta x_{j+1})}-(X_{j}^{2}-4X_{j}+2)e^{-X_{j}} \\
& X_{j}=L_{x}[1-\cos(\frac{\pi j}{2N_{x}})] \\
& \Delta x_{j}=X_{j}-X_{j-1} \\
\end{split}
\end{equation}
A comparison of grids is shown in Fig.\ref{Fig.2-cGrid}
\begin{figure}[htbp]
\centering
{
    \includegraphics[width=9cm]{./2/c_grid.png}
}
\caption{Comparison of non-uniform grid and uniform grid ($N_{x}=10$)} \label{Fig.2-cGrid}
\end{figure}

Plot the exact and numerical steady solutions for $N_{x}=10,20$ as Fig.\ref{Fig.2-c}.

\begin{figure}[htbp]
\centering
{
    \includegraphics[width=9cm]{./2/c.png}
}
\caption{Results of FTCS with different non-uniform mesh grids} \label{Fig.2-c}
\end{figure}

Compare Fig.\ref{Fig.2-b} and Fig.\ref{Fig.2-c} we can conclude using more (non-uniform grid) points in the region where solution changes rapidly (e.g., $0\leq x \leq 5$) will reduce the error.

\subsection{Coordinate Transformation}
Transform the differential equation a new coordinate system using the transform $\xi_{j}=\cos^{-1}(1-X_{j}/L_{x})$.
\begin{equation}
\begin{split}
&   X_{j}=L_{x}[1-\cos(\frac{\pi j}{2N_{x}})] \\
&   \xi_{j}=\frac{\pi j}{2N_{x}} \\
\end{split}
\end{equation}
Based on the chain rule of derivatives, the transformation is conducted as
\begin{equation}
Define: \ \  A = \frac{\partial }{\partial x},\ thus, \ \  A= (\frac{\partial \xi}{\partial x}) \ \frac{\partial }{\partial \xi}
\end{equation}

\begin{equation}
Define: \ \ B=\frac{\partial ^{2}}{\partial x \partial \xi}=\frac{\partial}{\partial x}(\frac{\partial }{\partial \xi})
,\ thus,\ B=A(\frac{\partial }{\partial \xi})= (\frac{\partial \xi}{\partial x}) \ \frac{\partial }{\partial \xi} (\frac{\partial }{\partial \xi})=(\frac{\partial \xi}{\partial x})(\frac{\partial ^{2}}{\partial \xi^{2}})
\end{equation}

\begin{equation}
\frac{\partial ^{2}}{\partial x^{2}} =\frac{\partial A}{\partial x}=\frac{\partial }{\partial x}(\frac{\partial \xi}{\partial x} \ \frac{\partial }{\partial \xi} ) = \frac{\partial^{2}\xi}{\partial x^{2}}\frac{\partial }{\partial \xi}+\frac{\partial \xi}{\partial x}\frac{\partial ^{2}}{\partial x \partial \xi}=(\frac{\partial ^{2}\xi}{\partial x^{2}})(\frac{\partial }{\partial \xi})+(\frac{\partial \xi}{\partial x})^{2}(\frac{\partial ^{2}}{\partial \xi^{2}}) \label{Eq.partial}
\end{equation}
then,
\begin{equation}
\frac{\partial ^{2}T}{\partial x^{2}}=
\frac{\partial ^{2}\xi}{\partial x^{2}}\frac{\partial T}{\partial \xi}+(\frac{\partial \xi}{\partial x})^{2}\frac{\partial ^{2}T}{\partial \xi^{2}}
\end{equation}
For $\xi=\cos^{-1}(1-x/L_{x})$, we have
\begin{equation}
\frac{\partial \xi}{\partial x}=\frac{1}{\sqrt{2x L_{x}-x^{2}}}
\end{equation}
\begin{equation}
\frac{\partial ^{2}\xi}{\partial x^{2}}=\frac{-(L_{x}-x)}{\sqrt{(2x L_{x}-x^{2})^{3}}}
\end{equation}

the heat equation in $(\xi, t)$ coordinate system is
\begin{equation}
\frac{\partial T}{\partial t}=\alpha [\frac{\partial ^{2}\xi}{\partial x^{2}}\frac{\partial T}{\partial \xi}+(\frac{\partial \xi}{\partial x})^{2}\frac{\partial ^{2}T}{\partial \xi^{2}}]+S[L_{x}(1-\cos \xi)]
\end{equation}
discretization form is
\begin{equation}
    \begin{split}
\frac{T_{j}^{n+1}-T_{j}^{n}}{\Delta t}= & \alpha[\frac{-(L_{x}-X_{j})}{\sqrt{(2X_{j} L_{x}-X_{j}^{2})^{3}}}
\frac{T_{j+1}^{n}-T_{j-1}^{n}}{2\Delta \xi}+
\frac{1}{(2X_{j} L_{x}-X_{j}^{2})}\frac{T_{j+1}^{n+1}-2T_{j}^{n}+T_{j}^{n-1}}{(\Delta \xi)^{2}}
]  \\
&    +S[L_{x}(1-\cos \xi_{j})] \\
    \end{split}
\end{equation}

The numerical results is shown in Fig.\ref{Fig.2-d}.
\begin{figure}[htbp]
\centering
{
    \includegraphics[width=9cm]{./2/d.png}
}
\caption{Results of FTCS with different uniform mesh grids (after coordinate transformation)} \label{Fig.2-d}
\end{figure}
It should be pointed out that Fig.\ref{Fig.2-d} looks much like Fig.\ref{Fig.2-c}.
The difference is that here the calculation of $T$ is based on a uniform grid $\xi$,
while in Problem 2(3) calculation is on a non-uniform grid $x$.

\subsection{Crank-Nicolson Method}
Using Crank-Nicolson method for time advancement,
\begin{equation}
\frac{T_{i}^{n+1}-T_{i}^{n}}{\Delta t}=\alpha
\frac{{\frac{1}{2}(T_{i+1}^{n+1}+T_{i+1}^{n})-2\cdot \frac{1}{2}(T_{i}^{n+1}+T_{i}^{n})}+\frac{1}{2}(T_{i-1}^{n+1}+T_{i-1}^{n})}
{(\Delta x)^{2}}+S(X_{i})
\end{equation}
then, we have
\begin{equation}
\frac{\alpha \Delta t}{2(\Delta x)^{2}}T_{i-1}^{n+1}-[1+\frac{\alpha \Delta t}{(\Delta x)^{2}}]T_{i}^{n+1}+\frac{\alpha \Delta t}{2(\Delta x)^{2}}T_{i+1}^{n+1}=
-T_{i}^{n}-\frac{\alpha \Delta t}{2(\Delta x)^{2}}(T_{i+1}^{n}-2T_{i}^{n}+T_{i-1}^{n})-S(X_{i})\Delta t \label{Eq.CN-full}
\end{equation}
Define
\begin{equation}
    \begin{split}
& A = \frac{\alpha \Delta t}{2(\Delta x)^{2}} \\
& B = -[1+\frac{\alpha \Delta t}{(\Delta x)^{2}}] \\
& C = \frac{\alpha \Delta t}{2(\Delta x)^{2}} \\
& K_{i} = -T_{i}^{n}-\frac{\alpha \Delta t}{2(\Delta x)^{2}}(T_{i+1}^{n}-2T_{i}^{n}+T_{i-1}^{n})-S(X_{i})\Delta t \\
    \end{split}
\end{equation}
then, Eq.\ref{Eq.CN-full} is denoted as
\begin{equation}
AT_{i-1}^{n+1}+BT_{i}^{n+1}+CT_{i+1}^{n+1} = K_{i}
\end{equation}
The Thomas algorithm will be employed to solve above tridiagonal systems of equations.

The numerical results is shown in Fig.\ref{Fig.2-e}.
\begin{figure}[htbp]
\centering
{
    \includegraphics[width=9cm]{./2/e.png}
}
\caption{Results of Crank-Nicolson method on uniform mesh grids} \label{Fig.2-e}
\end{figure}


\subsection{Discussion}

For each method mentioned above, given grid points $N_{x}=20$.

The convergence criterion is
\begin{equation}
\frac{|T(t)-T(t-2)|}{\sum |T(t)|} \leq 10^{-7}
\end{equation}

\begin{enumerate}
  \item Find the maximum time step required for stable solutions.
          \begin{table}[htbp]
          \centering
          \caption{Comparison of different methods}
          \begin{tabular}{|l|c|c|c|c|}
          \hline
           & FTCS & Non-uniform grid & coordinate transform & Crank-Nicolson \\
          \hline
          $(\Delta t)_{\max}$ & 0.51 & 0.006 & 0.009 & $\ge 1.0$ \\
          \hline
          iteration  & 80   & 29460 & 20656 & 290 \\
          \hline
          \end{tabular}
          \end{table}

  \item Plot transient soulutions at $t=2$ and $t=10$ as shown in Fig.\ref{Fig.2-time}.
          \begin{figure}[htbp]
            \centering
            \subfigure[$t=2$]
            {
                \includegraphics[width=7cm]{./2/t2.png}
            }
            \subfigure[$t=10$]
            {
                \includegraphics[width=7cm]{./2/t10.png}
            }
            \caption{Comparison accuracy of different methods at $t=2$ and $t=10$} \label{Fig.2-time}
            \end{figure}
\end{enumerate}



\begin{itemize}
    \item We compare different methods at $t=2$, at least 2 time steps are required, thus, we set $\Delta t=1.0$ for Crank-Nicolson. Actually, $\Delta t$ can be much larger than 1.0 for Crank-Nicoslon scheme since Cranki-Nicolson scheme is unconditionally stable.
    \item FTCS scheme shows the largest discrepancy among all the test methods under unsteady condition, because the error is $O((\Delta x)^{2},\Delta t)$.
    \item Using non-uniform grid, i.e., put more mesh points in the region where solutions change rapidly, will help reduce the error.
\end{itemize}



\newpage
\section{2D Steady-state Heat Equation}
Solve steady-state 2D heat conduction  equation in unit square.

Boundary condition:
\begin{itemize}
  \item at $x=0$, \ $T=0$.
  \item at $x=1$, \ $T=0$.
  \item at $y=0$, \ $\frac{\partial T}{\partial y}=0$.
  \item at $y=1$, \ $T=\sin (\pi x)$.
\end{itemize}

The analytical solution can be obtained by variable separation
\begin{equation}
T_{exact} = \frac{\cosh(\pi y)}{\cosh(\pi)} \sin (\pi x)
\end{equation}

2D heat equation is
\begin{equation}
\frac{\partial T}{\partial t} = \alpha (\frac{\partial^{2} T}{\partial x^{2}}+\frac{\partial^{2} T}{\partial y^{2}})
\end{equation}
steady state can be obtained when $t \rightarrow \infty $ thus  $\partial T / \partial t = 0$.

The convergence criterion for the solution to reach steady state is
\begin{equation}
\frac{|T(t)-T(t-100)|}{\sum |T(t)|} \leq 10^{-7}
\end{equation}

Discretization of heat equation:
\begin{equation}
\frac{T_{i,j}^{n+1}-T_{i,j}^{n}}{\Delta t}=\alpha [ \frac{T_{i+1,j}^{n}-2T_{i,j}^{n}+T_{i-1,j}^{n}}{(\Delta x)^{2}} +\frac{T_{i,j+1}^{n}-2T_{i,j}^{n}+T_{i,j-1}^{n}}{(\Delta y)^{2}} ]
\end{equation}

Discretization of Neumann boundary condition:
\begin{equation}
(\frac{\partial T}{\partial y})_{j=1}=\frac{-3T_{i,1}+4T_{i,2}-T_{i,3}}{\Delta y}=0
\end{equation}
thus,
\begin{equation}
T_{i,1}=\frac{4T_{i,2}-T_{i,3}}{3}
\end{equation}



Here is the main code snippets:
\lstinputlisting[language=Fortran]{./codes/3_snips.f90}

The temperature field shown in Fig.\ref{Fig.3}.
\begin{figure}[htbp]
\centering
\subfigure[Analytical solution]
{
    \includegraphics[width=7cm]{./3/3-exact.png}
}
\subfigure[Numerical solution (Mesh:$20\times 20$)]
{
    \includegraphics[width=7cm]{./3/3.png}
}
\caption{Steady-state 2D heat conduction } \label{Fig.3}
\end{figure}

Comparison of center temperature with the exact solution shown in Fig.\ref{Fig.3compare}
\begin{figure}[htbp]
\centering
\subfigure[Temperature at $y=0.5$ plane]
{
    \includegraphics[width=7cm]{./3/yHalf.png}
}
\subfigure[Temperature at $x=0.5$ plane]
{
    \includegraphics[width=7cm]{./3/xHalf.png}
}
\caption{Comparison of center temperature with analytical solution} \label{Fig.3compare}
\end{figure}


\newpage
\section{MacCormack Method for Burgers Eequation}
Solve linearized  Burgers equation $u_{t}+cu_{x}=\mu u_{xx}$, $c=0.5$, $\mu=0.02$.

Initial condition $u(x,0)=0$, $0\leq  x \leq 1$.

Boundary condition $u(0,t)=100$, $u(1,t)=0$.

Choose $\Delta x=0.02$ and $\Delta t=0.001$.

The exact solution for steady-state is
\begin{equation}
u(x)=100\frac{1-\exp[25(x-1)]}{1-\exp(-25)}
\end{equation}

Use MacCormack method, discretization is
\begin{equation}
    \begin{split}
& Predictor:  \overline{u_{j}^{n+1}}=u_{j}^{n}-c\frac{\Delta t}{\Delta x}(u_{j+1}^{n}-u_{j}^{n})+\frac{\mu \Delta t}{(\Delta x)^{2}}(u_{j+1}^{n}-2u_{j}^{n}+u_{j-1}^{n}) \\
& Corrector:  u_{j}^{n+1}=\frac{1}{2}[u_{j}^{n}+\overline{u_{j}^{n+1}}-c\frac{\Delta t}{\Delta x}(\overline{u_{j}^{n+1}}-\overline{u_{j-1}^{n+1}})+\frac{\mu \Delta t}{(\Delta x)^{2}}(\overline{u_{j+1}^{n+1}}-2\overline{u_{j}^{n+1}}+\overline{u_{j-1}^{n+1}})] \\
    \end{split}
\end{equation}

Here is the main code snippets:
\lstinputlisting[language=Fortran]{./codes/4_2_snips.f90}

The results shown in Fig.\ref{Fig.4}.
\begin{figure}[htbp]
\centering
{
    \includegraphics[width=9cm]{./4/4.png}
}
\caption{MacCormack method for linearized Burgers equation ($N_{x}=50$)} \label{Fig.4}
\end{figure}



%----------------------------------------------------------------------------------------------------------------------------------------------------------
%\bibliography{Bib/citations}
%----------------------------------------------------------------------------------------------------------------------------------------------------------
\end{CJK} % 结束中文环境
\end{document}
